\documentclass[runningheads]{llncs}

%\documentclass{article}
\usepackage{tikz}
\usepackage{amsmath}
\usepackage{amssymb}
\usepackage[]{algorithm2e}
%\usepackage{algorithm}
\usepackage[noend]{algpseudocode}
\usepackage{cite}
\usepackage{etoolbox}
\usepackage{mathtools}

%\usepackage{amsthm,pifont}

\usepackage[utf8]{inputenc}
%\newtheorem{theorem}{Theorem}
%\newtheorem{corollary}{Corollary}[theorem]

\newcommand{\cP}{{\mathcal P}}
\newcommand{\cF}{{\mathcal F}}
\newcommand{\cG}{{\mathcal G}}
\newcommand{\cO}{{\mathcal O}}
\newcommand{\cL}{{\mathcal L}}
\newcommand{\cR}{{\mathcal R}}



%\theoremstyle{plain}% default
%\newtheorem{thm}{Theorem}[section]
%\newtheorem{corollary}{Corollary}[thm]


\begin{document}
%\title{Neighborhood-based Reductions and Cuts for Signed Graphs}
%\author{Christopher Muir$^1$, Tony K. Rodriguez$^{1,2}$, James A. Ostrowski$^1$}
%\date{}
%\maketitle
%\begin{abstract}

%\end{abstract}



Consider a linear program $LP(A,b,c)$ in the following form:
\begin{align}
  \max \ & c^t x \\
  \mbox{s.t. } & Ax \leq b\\
  & x \geq 0,
  \end{align}
\noindent with  $x$ and  $c$ are in $\mathbb{R}^n$, $b$ is in $\mathbb{R}^m$ and
$A \in \mathbb{R}^{m \times n}$.

Let $\cP = \{P_1,\ P_2,\ \ldots,\ P_k\}$ be partition of the decision variables. We define the restricted
linear program, $RLP(A,B,c,\cP)$ to be:

\begin{align}
  \max \ & c^t x \\
  \mbox{s.t. } & Ax \leq b\\
  & x_i = x_i \ \ \forall\ i,j \ \in P_l,\  l= 1,\ \ldots,\ k \label{cons:equal}\\
  & x \geq 0,
  \end{align}

\noindent Obviously can be written in such a way as to aggregate variables that
belong in the same partition. Such an aggregation will yield a formulation with
only $k$ variables and (potentially) fewer constraints. It is easier, however, to
discuss the proposed methods using the above formulation as both, as stated,  $LP(A,b,c)$ and
$RLP(A,B,c,\cP)$ exist in the same dimension.

For this work we are primarily focused on partitions $\cP$ such that the optimal
solution to $LP(A,b,c)$ is equal to that of $RLP(A,B,c,\cP)$. We note that
techniques have been developed that attempt find such $\cP$, most notably those
that seek to exploit symmetry and its generalization, equitable partitions. In
such cases, the benefit of  $RLP(A,B,c,\cP)$ is clear, preprocessing will result
in a smaller formulation that will likely be easier to solve. However, the
solutions returned by solving the restricted problem will necessarily be in the
interior of the optimal face (unless $\cP$ consists only of singletons). In many
cases it is desirable to find an optimal vertex to $LP(A,b,c)$. Various
crossover methods have been developed to map an interior solution to an optimal
vertex, but such methods tend to be very computationally taxing. We show here
that mapping a solution from restricted problems generated by symmetry and/or
equitable partitions to optimal vertices are
considerably easier than the more general crossover methods.


\section{Background: Symmetry and Equitable Partitions}



The symmetry group of $LP(A,b,c)$ is defined to
be permutations that map feasible solutions to the LP to feasible solutions with
the same objective value. Let $\cG$ represent the symmetry group. A problem's
symmetry group can be used to define equivalence classes on both the sets of
variables as well as the set of feasible solutions. As permutations act on
vectors, it is most natural to think of how symmetries act on solution vectors.
We say two solutions are equivalent if there exists a permutation in $\cG$ that
maps one solution to the other. We call the set of all solutions equivalent to a
solution $x$ an {\em orbit} of $x$ with respect to $\cG$. We use
$\mbox{Orb}(x,\ \cG)$ to represent the orbit of $x$ with respect to $\cG$. We
overload the $\mbox{Orb}$ notation to act on variables as well as vectors. The
orbit of variable $x_i$ with respect to $\cG$ are those variables that are
equivalent to $x_i$. We have that $j \in \mbox{Orb}(i,\cG)$ if and only if $e_j
\in \mbox{Orb}(e_i,\cG)$. The variable orbits can be used to define a natural
parition of the variables as $i,\ j \in \mbox{Orb}(h,\cG)$ implies that $h, j
\in \mbox{Orb}(i,\cG)$. We let $\cO= \{O_1,\ldots,\ O_k\}$ represent the {\em orbital
partition} of the variables. 

\begin{theorem} \label{thm:set_equal}
Let $\cG$ be the symmetry group acting on a linear program and let $\cO$ be the
orbital partition with respect to $\cG$. The optimal solution
value of $LP(A,b,c)$ is equal to the optimal solution value of $RLP(A,b,c, \cO)$
\end{theorem}


\begin{corollary}
As a result of these fixings, an aggregated LP with the same optimal objective
value can be written using $|\cO^V|$ many variables and $|\cO^C|$ many constraints.
  \end{corollary}

We say that partition $\cP^1$ is a {\em refinement} of partition $\cP^2$ if for
all $P_i^1 \in \cP^1$ there exists a $P_j^2 \in \cP^2$ with $P_i^1 \subseteq
P_j^2$. If $\cP^1$ is a refinement of $\cP^2$, then we have that the LP solution
to $RLP(A,B,c,\cP_1)$ is no smaller than $RLP(A,B,c,\cP_2)$. Thus, for any
partition that is a refinement of the problem's orbital partition $\cO$,
Theorem~\ref{thm:set_equal} ensures that $RLP(A,b,c,\cP_1)$ has the same
objective value of $LP(A,b,c)$.

{\em Dimensions of RLP polyhedra:}

Note that the polyhedra defined by the feasible region of both $LP(A,b,c)$ and
$RLP(A,b,c,\cP)$ are both in $\mathbb{R}^n$. However, while the feasible
region of $LP(A,b,c)$ can be fully dimensional, that of $RLP(A,b,c,\cP)$ is not
(except for trivial partitions). Indeed, it is easy to see that the set of
equalities defined in~\eqref{cons:equal} have a rank of $n - |\cP|$.

{\em Hierarchy of  optimal restrictions:}

We new define a hierarchy of restrictions that are all guaranteed to share the
same optimal objective value (that of the original LP). First, we define the {\em
  stabilizer} of a group $\cG$ with respect to an element $v$, $stab(v,\cG)$ to
be:

$$stab(v,g) := \{ \pi \in \cG\ | \ \pi(v) = v\}.$$ 

\noindent That is, it consists of permutations in $\cG$ that fix the element
$v$. The stabilizer of an element is still a group (in fact a subgroup of
$\cG$). We define $stab^0$ to just be $\cG$ and recursively define $stab^k(\cG)$
to be $stab( k , stab^{k-1}(\cG))$. As every $stab^k$ is a group, it can be used
to generate an orbital partition $\cO^k$ (with respect to $stab^k$). As $stab^{k+1}$
is a subset of $stab^k$, we have that $\cO^{k+1}$ is a refinement of $\cO^k$.
Thus, $\{dim(RLP(A,b,c,\cO^k)\}_{ k }$ is a nondecreasing
sequence and that $\cO^n$ consists of only singletons.


Generalize notions to equitable partitions.

\section{Required Theory}

Things to show(?)
\begin{itemize}
\item Active constraints (bounds) using $\cO^k$ are active in $\cO^{k+1}$: this isn't
  super obvious in the aggregated model, but is in the restricted version(?)
  \item When one constraint becomes active, so are all others in its respective
    orbit, defined by *some* stabilizer. We have to be a bit careful in showing
    this. Crossing over  from $\cO^k$ to $\cO^{k+1}$, the stabilizer we care
    about is $stab^{k+1}$

    \end{itemize}

    Things to consider in the algorithm:
    \begin{itemize}
    \item Three types of pivots, degenerate, those that hit a singleton
      constraint orbit, and those that hit a singleton constraint, and those
      that hit a nontrivial constraint orbit.
      \begin{itemize}
      \item Degenerate: We can drop the corresponding equality constraint and
        move on.
      \item Singleton: We add one linearly independent active constraint
        \item Non trivial: we add possibly more than one linearly independent
          active constraint. To determine this, we only need to compute a
          maximal set of linearly independent constraints for that orbit
          (proof). Can we determine this using the symmetry group instead?
      \end{itemize}
      \end{itemize}
      

{\em Crossover}

We use the above facts to design a crossover strategy that will take a vertex
solution to $RLP(A,b,c,\cO^0\}$ and  lift it to a vertex solution to
  $RLP(A,b,c,\cO^n) = LP(A,b,c)$. This will be done iteratively by lifting each
vertex solution of  $RLP(A,b,c,\cO^i\}$  to a vertex solution to
  $RLP(A,b,c,\cO^{i+1})$ 



{\em Not entirely sure why I put all of this here... Pick and choose...}


{\bf Orbits of Vertices}

Similar to variables, permutations can act on vectors. We can use symmetry to
compute how many equivalent vertices there are in a polyhedron. First, recall
the Orbit-Stabilizer Theorem (OST):

\begin{theorem}
For $x \in \mathbb{R}^n$ and group $\cG$, we have $|\mbox{Orb}(x,\cG)| = \frac{ |\cG|}{|\mbox{Stab}(x,\cG)|}$.
  \end{theorem}

The intuition behind the OST is as follows. Some
permutations in $\cG$ map $x$ to itself. These permutations are those that make
up $\mbox{Stab}(x,\cG)$. By somehow dividing $\cG$ by these permutations we will
be left with permutations that map $x$ to a unique vector.

Using OST, then, for a vertex $x^i$, the set of equivalent vertices contains
$\frac{ |\cG|}{|\mbox{Stab}(x^i,\cG)|}$. Let $B^i$ and $NB^i$ be the set of
basic and nonbasic variables associated with vertex $x^i$. Because $NB^i$ is
enough to uniquely determine $x^i$, we have that $\mbox{Stab}(x^i,\cG) =
\mbox{Stab}(NB^i,\cG)$. Also, as $\mbox{Stab}(S,\cG) =
\mbox{Stab}(\overline{S},\cG)$ for any set $S$, we also have that $\mbox{Stab}(x^i,\cG) =
\mbox{Stab}(B^i,\cG)$. Thus, we can compute the size of the orbit of a given
vertex by just using the basis information (the actual solution doesn't have to
be known). 

In addition the the number of equivalent vertices, symmetry information can be
used to identify relationships between active and inactive constraints. Consider
the following theorem:

\begin{theorem}
For vertex $x^i$, any constraint orbit generated with respect to
$\mbox{Stab}(NB^i,\cG)$ contains either all binding constraints or all inactive constraints. 
\end{theorem}

Note that the above theorem relies on the stabilizer of $x^i$, not the whole
group. Indeed, the size of the orbits with respect to the stabilizer will be no larger than
those with respect to $\cG$. The solution to the aggregated solution can be used
to give insight on the set of active constraints for the corresponding vertices.


\begin{theorem}
For a vertex $x^i$, let $\overline{x}$ be the average of all vertices in
$\mbox{Orb}(x^i, \cG)$. Then, for any constraint orbit $O^C$, either all
constraints in that orbit are binding or none of them are. 
  \end{theorem}

\begin{corollary}
For a vertex $x^i$, let $\overline{x}$ be the average of all vertices in
$\mbox{Orb}(x^i, \cG)$. If a constraint is binding for $\overline{x}$, than the
constraint is binding for all vertices in $\mbox{Orb}(x^i, \cG)$.
  \end{corollary}







\begin{algorithm}[H]
 \KwData{Aggregated solution $x^0$, set of active constraints $C$, coarsest equitable partition $\cP_0$}
 \KwResult{A vertex solution to the original LP}
% initialization\;
$\cP_0 \eqqcolon \cP$\\ 
 \While{$|\cP| \neq m + n$}{
 	$\cP \eqqcolon \bar{\cP}$\;
 	$\mbox{iso}(C, \cP)  = \{\cP^1, \dots, \cP^k\}$, a new partition\;
 	\textbf{Refine($\{P^1, \dots, P^k\}$)}$\eqqcolon \cP$\;
  $\{(x_i, x_j)|\cP^i, \cP^j \subseteq \cP^k \in \bar{\cP}, \ \forall \cP^j \subseteq \cP^k, j \neq i, \ \forall \cP^k \in \bar{\cP}\} \eqqcolon \cL$\;
  Create $RLP(A,b,c,\cP)$\;
\For{$(x_i, x_j) \in L$}{
  $x_i - x_j \eqqcolon z$\;
  $\max z$, alter objective of $RLP(A,b,c,\cP)$\;
  Relax $x_i = x_j$\;
  \textbf{Simplex}($RLP(A,b,c,\cP)$) $\eqqcolon \cR$\;
  $C \cup \cR \eqqcolon C$\;
  }

 }
 \caption{Orbital Crossover}
 

\end{algorithm}

See "cite Grohe" for the subroutine \textbf{Refine} and note that the \textbf{Simplex} subroutine is performing the \textit{Simplex Algorithm} and returning the active constraints of $LP(A,b,c)$ based of the active constraints in $RLP(A,b,c,\cP)$.

\end{document}
