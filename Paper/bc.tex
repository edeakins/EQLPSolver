% !TEX TS-program = pdflatex
% !TEX encoding = UTF-8 Unicode

% This is a simple template for a LaTeX document using the "article" class.
% See "book", "report", "letter" for other types of document.

\documentclass[11pt]{article} % use larger type; default would be 10pt

\usepackage[utf8]{inputenc} % set input encoding (not needed with XeLaTeX)

%%% Examples of Article customizations
% These packages are optional, depending whether you want the features they provide.
% See the LaTeX Companion or other references for full information.

%%% PAGE DIMENSIONS
\usepackage{geometry} % to change the page dimensions
\geometry{a4paper} % or letterpaper (US) or a5paper or....
% \geometry{margin=2in} % for example, change the margins to 2 inches all round
% \geometry{landscape} % set up the page for landscape
%   read geometry.pdf for detailed page layout information

\usepackage{mathtools}
\usepackage[short]{optidef}
\DeclarePairedDelimiter{\ceil}{\lceil}{\rceil}
\usepackage{graphicx} % support the \includegraphics command and options

% \usepackage[parfill]{parskip} % Activate to begin paragraphs with an empty line rather than an indent

%%% PACKAGES\usepackage{amsmath}
\usepackage{amsfonts}
\usepackage{amssymb}
\usepackage{amsmath}
\usepackage{booktabs} % for much better looking tables
\usepackage{array} % for better arrays (eg matrices) in maths
\usepackage{paralist} % very flexible & customisable lists (eg. enumerate/itemize, etc.)
\usepackage{verbatim} % adds environment for commenting out blocks of text & for better verbatim
\usepackage{subfig} % make it possible to include more than one captioned figure/table in a single float
% These packages are all incorporated in the memoir class to one degree or another...

%%% HEADERS & FOOTERS
\usepackage{fancyhdr} % This should be set AFTER setting up the page geometry
\pagestyle{fancy} % options: empty , plain , fancy
\renewcommand{\headrulewidth}{0pt} % customise the layout...
\lhead{}\chead{}\rhead{}
\lfoot{}\cfoot{\thepage}\rfoot{}
\newcolumntype{L}{>{$}l<{$}} % math-mode version of "l" column type

%%% SECTION TITLE APPEARANCE
%\usepackage{sectsty}
%\allsectionsfont{\sffamily\mdseries\upshape} % (See the fntguide.pdf for font help)
% (This matches ConTeXt defaults)

%%% ToC (table of contents) APPEARANCE
\usepackage[nottoc,notlof,notlot]{tocbibind} % Put the bibliography in the ToC
\usepackage[titles,subfigure]{tocloft} % Alter the style of the Table of Contents
\renewcommand{\cftsecfont}{\rmfamily\mdseries\upshape}
\renewcommand{\cftsecpagefont}{\rmfamily\mdseries\upshape} % No bold!
\usepackage{setspace}
\usepackage{float}
\usepackage{tikz}
\usepackage{algorithm2e}
\usepackage{caption}
\usepackage{indentfirst}
\newcommand{\cP}{{\cal P}}
\newcommand{\cA}{{\cal A}}
\newcommand{\cC}{{\cal C}}
\newcommand{\cB}{{\cal B}}

%%% END Article customizations

%%% The "real" document content comes below...


\title{LP Aggregation and Higher Dimensional Basis Construction}
\author{Ethan Deakins}
 
 \begin{document}
 	\maketitle
 	
 	\section*{Background}
 	The following document aims to lay out the steps of a basis construction algorithm for an aggregated LP.  For instance, using equitable partitions, if we are able to reduce the dimension of an LP with $m = 10$ and $n = 9$ to $m = 2$ and $n = 1$, then what steps must we take to construct a basis in the original dimensions.  From here on, we will be using a problem instance called "codbt02.mps" to lay the out the algorithm.
 	
 	\section*{Algorithm Sketch}
 	The instance mentioned previously has the following form 
 	
	 \begin{mini!}
	 	{}{x_0 + x_1 + x_2 + x_3 + x_4 + x_5 + x_6 + x_7 + x_8}{}{}
	 	\addConstraint{x_0 + x_1 + x_2 + x_3 + x_6}{\geq 1}{}
	 	\addConstraint{x_0 + x_1 + x_2 +  x_4 + x_7}{\geq 1}{}
	 	\addConstraint{x_0 + x_1 + x_2 +  x_6 + x_8}{\geq 1}{}
	 	\addConstraint{x_0 + x_3 + x_4 +  x_5 + x_6}{\geq 1}{}
	 	\addConstraint{x_1 + x_3 + x_4 +  x_5 + x_7}{\geq 1}{}
	 	\addConstraint{x_2 + x_3 + x_4 +  x_5 + x_8}{\geq 1}{}
	 	\addConstraint{x_0 + x_3 + x_6 +  x_7 + x_8}{\geq 1}{}
	 	\addConstraint{x_1 + x_4 + x_6 +  x_7 + x_8}{\geq 1}{}
	 	\addConstraint{x_2 + x_5 + x_6 +  x_7 + x_8}{\geq 1}{}
	 	\addConstraint{x_0 + x_1 + x_2 + x_3 + x_4 + x_5 + x_6 + x_7 + x_8}{\geq 2}{}
	 	\addConstraint{\bar{x}_i}{\leq 1, \quad}{i = 0, 1, \hdots, 8}
	 	\addConstraint{\bar{x}_i}{\geq 0, \quad}{i = 0, 1, \hdots, 8}
	 \end{mini!}	
 
	Constraint (1k) is added to the instance due to the fact that, when aggregated, the aggregated solution to the LP is a vertex.  This constraint prohibits this, and thus allows for this problem to be used as an example.
	
\begin{equation}
 	\begin{pmatrix}
 		\begin{array}{ccccccccc|c}
 			1 & 1 & 1 & 1 & 0 & 0 & 1 & 0 & 0 & 1 \\ 
 			1 & 1 & 1 & 0 & 1 & 0 & 0 & 1 & 0 &  1 \\ 
 			1 & 1 & 1 & 0 & 0 & 1 & 0 & 0 & 1 & 1 \\ 
 			1 & 0 & 0 & 1 & 1 & 1 & 1 & 0 & 0 & 1 \\ 
 			0 & 1 & 0 & 1 & 1 & 1 & 0 & 1 & 0 & 1 \\ 
 			0 & 0 & 1 & 1 & 1 & 1 & 0 & 0 & 1 & 1 \\ 
 			1 & 0 & 0 & 1 & 0 & 0 & 1 & 1 & 1 & 1 \\ 
 			0 & 1 & 0 & 0 & 1 & 0 & 1 & 1 & 1 & 1 \\ 
 			0 & 0 & 1 & 0 & 0 & 1 & 1 & 1 & 1 & 1 \\ 
 			\hline
 			1 & 1 & 1 & 1 & 1 & 1 & 1 & 1 & 1 & 2 \\ 
 			\hline
 			1 & 1 & 1 & 1 & 1 & 1 & 1 & 1 & 1 & \infty
 		\end{array}
 	\end{pmatrix}
 \end{equation}
 
	 The matrix format of the instance is given in (2).  The horizontal and vertical bars represent the aggregated constraints and variables. As is seen from the matrix in (2), the coarsest equitable partition for (1) is
	 
	 \begin{equation}
	 	\mathcal{P}^1 = 
	 	\begin{pmatrix} 
	 	x_0 & x_1 & x_2 & x_3 & x_4 & x_5 & x_6 & x_7 & x_8 \\
	 	c_0 & c_1 & c_2 & c_3 & c_4 & c_5 & c_6 & c_7 & c_8 \\
	 	c_9
	 	\end{pmatrix} 
	 \end{equation}
	 
	 From $\mathcal{P}^1$ we have that the aggregated LP of (1) is  
 
 \begin{mini!}
 	{}{\bar{x}_0^1}{}{}
 	\addConstraint{9 \bar{x}_0^1 - \bar{s}_0^1}{= 2}
 	\addConstraint{5 \bar{x}_0^1 - \bar{s}_1^1}{= 1}
	\addConstraint{0 \leq }{\bar{x}_0^1, \bar{s}_0^1, \bar{s}_1^1}{\leq 1}
 \end{mini!}

	In (3), the variable $\bar{x}_0^1$ is the aggregated variable from (1) under the first coarsest equitable partition (superscript) where $x_0$ is the representative of that equitable partition (subscript).  Solving this aggregation results in the following solution: $\bar{x}_0^1 = \frac{2}{9}$, $\bar{s}_1^1 = \frac{1}{9}$, and $\bar{s}_0^1 = 0$ (non-basic).  Here, the variables denoted by $\bar{s}_i^j$ are the slack variables for the $i^{th}$ constraint under the $j^{th}$ coarsest equitable partition.  This solution gives the basis
	
	\begin{equation}
	\begin{pmatrix}
	\begin{array}{ccc|c}
		\bar{x}_0^1 & \bar{s}_0^1 & \bar{s}_1^1 \\
		\hline
		1 & -\frac{1}{9} & 0 & \frac{2}{9} \\
		0 & -\frac{5}{9} & 1 & \frac{1}{9} 
	\end{array}
	\end{pmatrix}
	\end{equation}
	
	  To construct the next basis, we must first come back to our equitable partition $\mathcal{P}^1$.  Now we will choose variable $x_0$ from (1) and give it a new color (isolation).  After having done this, we will recompute the coarsest equitable partition for this instance.  This results in
	
	\begin{equation}
	\mathcal{P}^2 = 
	\begin{pmatrix}
	\begin{array}{c|cccc}
	\cC_1 & x_1 & x_2 & x_3 & x_6 \\
	\cC_0 & x_0 &&& \\
	\cC_3 & x_4 & x_5 & x_7 & x_8 \\
	\cC_5 & c_4 & c_5 & c_7 & c_8 \\
	\cC_4 & c_0 &&& \\
	\cC_6 & c_1 & c_2 & c_3 & c_6 \\
	\cC_7 & c_9 &&&
 	\end{array}
	\end{pmatrix}
	\end{equation}
	
	where $x_i$ is the $i^{th}$ variable in (1) and $c_i$ is the $i^{th}$ constraint in (1). From this equitable partition we now can construct the next aggregated LP.  It has the form
	
	\begin{mini!}
		{}{\bar{x}_0^2 + 4 \bar{x}_1^2 + 4 \bar{x}_4^2}{}{}
		\addConstraint{2\bar{x}_0^2 + 3 \bar{x}_4^2 - \bar{s}_0^2}{= 1}
		\addConstraint{\bar{x}_0^2 + 3 \bar{x}_1^2 + \bar{x}_4^2 - \bar{s}_1^2}{= 1}
		\addConstraint{\bar{x}_0^2 + 4 \bar{x}_1^2 - \bar{s}_2^2}{= 1}
		\addConstraint{\bar{x}_0^2 -  \bar{x}_1^2 - \bar{r}_{01}}{= 0}
		\addConstraint{\bar{x}_0^2 -  \bar{x}_4^2 - \bar{r}_{04}}{= 0}
		\addConstraint{\bar{x}_0^2 + 4\bar{x}_1^2 + 4\bar{x}_4^2}{= {2}}
		\addConstraint{\bar{x}_i^2}{\leq 1, \quad}{i = 0, 1, 4}
		\addConstraint{\bar{x}_i^2}{\geq 0, \quad}{i = 0, 1, 4}
		%\addConstraint{\bar{r}_{01} = \bar{r}_{04}}{= 0}
	\end{mini!}

	Here, there are a few points to notice.  First, the active set from the previous aggregation have been brought forward, (6g).  Secondly new constraints have been added to show that, for the time being, The new aggregate variables should remain equal to each other, as they were all equal in the previous aggregation. 
	
	Now we may begin constructing the basis for this aggregation based on the solution from the previous aggregation.
	
	\begin{equation}
	\begin{pmatrix}
	\begin{array}{cccccccc|c}
	\bar{x}^2_0 & \bar{x}^2_1 & \bar{x}^2_4 & \bar{s}^2_0 & \bar{s}^2_1 & \bar{s}^2_4 & \bar{r}_{01} & \bar{r}_{04} \\
	\hline
	1 & 0 & 0 & 0 & 0 & 0 & -\frac{4}{9} & -\frac{4}{9} & \frac{2}{9} \\ 
	0 & 1 & 0 & 0 & 0 & 0 & \frac{5}{9} & 0 & \frac{2}{9} \\ 
	0 & 0 & 1 & 0 & 0 & 0 & 0 & \frac{5}{9} & \frac{2}{9} \\ 
	0 & 0 & 0 & 1 & 0 & 0 & -\frac{8}{9} & \frac{7}{9} & \frac{1}{9} \\ 
	0 & 0 & 0 & 0 & 1 & 0 & \frac{11}{9} & \frac{1}{9} & \frac{1}{9} \\
	0 & 0 & 0 & 0 & 0 & 1 & -\frac{16}{9} & \frac{4}{9} & \frac{1}{9} \\
	\end{array}
	\end{pmatrix}
	\end{equation}
	
	Now, we shall explain how this basis is constructed.  First, as mentioned previously, the active set from (4) comes forward to (7).  Doing this creates (7g).  Inspection will show that (7g) is just the disaggregated version of (4b) based on $\cP^2$. Next, (7b, 7c, 7d) are the disaggregated constraints of (4c) based on $\cP^2$.  Since (4c), was not active in the solution to (4), its disaggregated constraints in (7) will not be active either and will have slacks (for the time being).  Finally, (7e, 7f) are called linking constraints and ensure that all aggregated variables in (7) remain equal in the new basis.
	
	Next, take note that $\bar{x}_0^2, \ \bar{x}_1^2, \ \text{and} \ \bar{x}_4^2$ are all the aggregated variables from each variable class within $\mathcal{P}^2$.  Because each variable class in $\cP^2$ is a subclass of the singular variable class in $\cP^1$, and because $\bar{x}_0^1 = \frac{2}{9}$, we know that $\bar{x}_0^2, \ \bar{x}_1^2, \ \text{and} \ \bar{x}_4^2$ must also be equal to $\frac{2}{9}$.  
	
	Similarly, we know that all new slack variables for constraints that are not active in the basis in (8) must be equal to $\frac{1}{9}$ because $\bar{s}_1^1 = \frac{1}{9}$ in (5).  The logic is the same as the variables, where, since each of the constraints where these slacks are located are subclasses of the singular class in $\cP^1$, and because the constraint that results from this singular class is not active in (5), the same must be true for the constrains which come from the subclasses of $\cP^1$ in $\cP^2$.
	
	Essentially, we brought forward all information forward from LP$^{k - 1}$.  We know that the matrix for the basic variables will be nothing more that an identity matrix.  Now all we have to do is take the correct steps to accomplish this while determining the values for the columns associated with the linking variables $\bar{r}$.  As it turns out, finding the entries in these column is not too difficult.
	
	First, we will take care of row 0, to which we will assign constraint (7g).  To construct (7g) in(8), all we must do is take $\frac{1}{9}(4 \times (7\text{e}) + 4 \times (7\text{f}) + (7\text{g}))$.  In general the rows that corresponds to active constraints that are the result of a previous aggregation have the following form.  Given an aggregated LP based on $\cP^k$, a known set of active constraints $\cA^{k - 1}$, and a set of basic variables $\cB^{k - 1}$ we have
	
	\begin{equation}
	\text{row}_{i}^k \coloneqq \frac{|\cC_b^{k - 1}|\bar{x}_i^k - \sum_{i \in \cP^k}|\cC^k_i|\bar{r}_{il} = |\cC^{k - 1}_b|\beta}{|\cC_b^{k - 1}|}
	\end{equation}
	
	Where $\bar{x}_i^k$ is a previously basic variable from LP$^{k - 1}$.  row$_j^k$ is an aggregated constraint of $C_i^k \subseteq C_\alpha^{k - 1}, \ \alpha \in \cA^{k - 1}$.  $\bar{r}_{il}$ is a linking variable that maintains equality between aggregated variables from LP$^{k - 1}$ and new aggregate variables in LP$^k$ which were the result of $\cC_i^k \subseteq \cC_b^{k - 1}$ where $\bar{x}_i^k$ was the aggregate of $\cC_b^{k - 1}$, a class which aggregated to become a basic variable in $\cP^{k - 1}$.  Finally, $\beta$ is the value of $\bar{x}_i^k$ in LP$^{k - 1}$.
		
	Once the rows of LP$^k$ that are associated with the active set of LP$^{k - 1}$ are determined, then the rows that are associated with the new linking constraints become very easy to determine.  They are of the form 
	
	\begin{equation}
	\text{row}_{\bar{r}_{il}} \coloneqq \bar{x}_i^k - \sum_{i \in \cP^k}|\cC^k_i|\bar{r}_{il} + 1 =\beta
	\end{equation} 
	
	Here $\bar{x}_i^k$ is a new variable from $\cP^k$ who shared a common class with a basic variable from LP$^{k - 1}$. 
	
 \end{document}


